\section{Introduction}\label{Introduction}

The autonomous industry has seen instantaneous growth, and the positioning of vehicles serves as a
critical milestone for a wide range of applications, ranging from advanced driver-assistance systems
(ADAS) to autonomous driving. A vehicle should realize its position concerning the environment at
level 3 or above for autonomous driving, according to the Society of Automotive Engineers, where
at level 0, the driver has full control of the vehicle, and at level 5, the vehicle has full control for all
challenging environments [14]. Germany became the first to embrace Level 3 automated driving
technology, which signifies a significant step towards autonomous driving. With major
manufacturers like BMW and Mercedes offering Level 3 systems in the European market and Level
4 testing in progress, the automotive industry is rapidly advancing towards fully autonomous vehicles
[15]. The shift towards higher levels of automation not only improves the safety and efficiency of
the vehicle but also paves the way for a future where self-driving cars are the norm.

Additionally, artificial intelligence and sensor technology advancements are key components in
developing reliable autonomous navigation systems. These systems must be able to interpret and
respond to complex environments in real-time to ensure safe and efficient operation. Perception,
localization and mapping, path planning, and control are the four main components of autonomous
driving. With more ADAS systems coming into the picture, more AD functions are added to the
latest mass-production vehicle, requiring immense testing and verification. This process needs
ground truth data like vehicle position, velocity, number of objects, and more to play an important
role in defining the accuracy of these systems.

A few major challenges faced by the automotive industry.
1. Accuracy: Achieving this level of precision is essential, especially for applications like
autonomous driving and advanced navigation systems.
2. Positioning Under Bridges: GNSS signals can be weak or blocked under bridges,
necessitating reliance on IMU data and dead reckoning.
3. Positioning in Tunnels: Similar to under bridges, GNSS signals are typically unavailable in
tunnels. Accurate dead reckoning is crucial here.
4. Positioning in GNSS-Denied Environments for Extended Periods: Maintaining accuracy
over time without GNSS support is challenging due to the potential drift in IMU data.
5. Ground Truth for Motion Planning: Accurate positioning data is critical for lane detection
and lane-keeping systems in autonomous vehicles.
6. Reference Point for Localization Algorithms: High accuracy in vehicle localization is
necessary for safe and efficient navigation.
7. Vehicle Location in Complex Interchanges: Identifying precise vehicle location in complex
highway interchanges like stack or cloverleaf intersections, where multiple layers and
directions can confuse navigation systems.